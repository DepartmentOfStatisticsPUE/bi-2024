% Options for packages loaded elsewhere
\PassOptionsToPackage{unicode}{hyperref}
\PassOptionsToPackage{hyphens}{url}
%
\documentclass[
]{article}
\usepackage{amsmath,amssymb}
\usepackage{iftex}
\ifPDFTeX
  \usepackage[T1]{fontenc}
  \usepackage[utf8]{inputenc}
  \usepackage{textcomp} % provide euro and other symbols
\else % if luatex or xetex
  \usepackage{unicode-math} % this also loads fontspec
  \defaultfontfeatures{Scale=MatchLowercase}
  \defaultfontfeatures[\rmfamily]{Ligatures=TeX,Scale=1}
\fi
\usepackage{lmodern}
\ifPDFTeX\else
  % xetex/luatex font selection
\fi
% Use upquote if available, for straight quotes in verbatim environments
\IfFileExists{upquote.sty}{\usepackage{upquote}}{}
\IfFileExists{microtype.sty}{% use microtype if available
  \usepackage[]{microtype}
  \UseMicrotypeSet[protrusion]{basicmath} % disable protrusion for tt fonts
}{}
\makeatletter
\@ifundefined{KOMAClassName}{% if non-KOMA class
  \IfFileExists{parskip.sty}{%
    \usepackage{parskip}
  }{% else
    \setlength{\parindent}{0pt}
    \setlength{\parskip}{6pt plus 2pt minus 1pt}}
}{% if KOMA class
  \KOMAoptions{parskip=half}}
\makeatother
\usepackage{xcolor}
\usepackage[margin=1in]{geometry}
\usepackage{graphicx}
\makeatletter
\def\maxwidth{\ifdim\Gin@nat@width>\linewidth\linewidth\else\Gin@nat@width\fi}
\def\maxheight{\ifdim\Gin@nat@height>\textheight\textheight\else\Gin@nat@height\fi}
\makeatother
% Scale images if necessary, so that they will not overflow the page
% margins by default, and it is still possible to overwrite the defaults
% using explicit options in \includegraphics[width, height, ...]{}
\setkeys{Gin}{width=\maxwidth,height=\maxheight,keepaspectratio}
% Set default figure placement to htbp
\makeatletter
\def\fps@figure{htbp}
\makeatother
\setlength{\emergencystretch}{3em} % prevent overfull lines
\providecommand{\tightlist}{%
  \setlength{\itemsep}{0pt}\setlength{\parskip}{0pt}}
\setcounter{secnumdepth}{-\maxdimen} % remove section numbering
\ifLuaTeX
  \usepackage{selnolig}  % disable illegal ligatures
\fi
\usepackage{bookmark}
\IfFileExists{xurl.sty}{\usepackage{xurl}}{} % add URL line breaks if available
\urlstyle{same}
\hypersetup{
  pdftitle={Symulacje na potrzeby projektu z badań internetowych},
  pdfauthor={Maciej Beręsewicz},
  hidelinks,
  pdfcreator={LaTeX via pandoc}}

\title{Symulacje na potrzeby projektu z badań internetowych}
\author{Maciej Beręsewicz}
\date{}

\begin{document}
\maketitle

\section{Symulacje}\label{symulacje}

Każda grupa projektowa przesyła kod poszczególnych symulacji przez
moodle do \emph{11.05 (23:59)}.

Wymagania dot. pliku:

\begin{itemize}
\tightlist
\item
  plik powiniem nazywać się: \emph{grupa-NR} i mieć jedno z
  następujących rozszerzeń:
  \texttt{.R,\ .RMD,\ .QMD,\ .PY,\ .IPYNB,\ .JL},
\item
  plik powiniem zawierać komentarze abym mógł zrozumieć co się tam
  dzieje,
\item
  jeżeli plik wykorzystuje biblioteki należy je załadować na samym
  początku.
\item
  W razie pytań proszę o kontakt
\end{itemize}

\subsection{Symulacja 1}\label{symulacja-1}

\subsection{Symulacja 2}\label{symulacja-2}

\subsection{Symulacja 3}\label{symulacja-3}

\subsubsection{Informacje o symulacji}\label{informacje-o-symulacji}

\begin{itemize}
\tightlist
\item
  \(N=10,000\) -- wielkość populacji,
\end{itemize}

\subsubsection{Zmienne pomocnicze}\label{zmienne-pomocnicze}

\begin{itemize}
\tightlist
\item
  \(p=15\) -- liczba zmiennych \(X_p\) gdzie \(p-1\) zmiennych
  generowanych jest z rozkładu \(N(0,1)\) (pierwsza kolumna tej macierzy
  to stała, wyraz wolny). Poniżej przykład jak powinna wygladać ta
  macierz (liczby przykładowe.)
\end{itemize}

\[
X=
\begin{bmatrix}
1 & -0.234 & ... &  0.325 & \\
1 & 0.434 & ...  &  -0.244 & \\
... & ... & ...  & -1.24 & \\
1 & -0.532 & ...  & 1.45 & 
\end{bmatrix}
\]

\subsection{Zmienna celu}\label{zmienna-celu}

\begin{itemize}
\item
  Dwie zmienne celu \(Y_1\) i \(Y_2\)
\item
  Ciągła: \[
        Y_1 = 1 + \exp(3\sin(\mathbf{\beta}^T\mathbf{X}_i)) + X_{5i} + X_{6i} + \epsilon
        \] gdzie \(\mathbf{\beta}=(1,0,0,1/2,-1/2,1/2,-1/2,0,...,0)^T\)
\item
  Binarna:
  \[Y_2 \sim \text{Ber}(\pi_Y(X))$ with $\text{logit}(\pi_Y(X)) = \mathbf{\beta}^TX_i
        \] gdzie \(\mathbf{\beta}=(1,0,0,-1,1,-1,1,0,...,0)^T\).
\end{itemize}

\subsubsection{Próby}\label{pruxf3by}

\begin{itemize}
\tightlist
\item
  \(S_A\) -- próba nielosowa (nieprobabilistyczna),
\item
  \(S_B\) -- próba losowa (probabilistyczna).
\end{itemize}

\subsubsection{Selekcja do próby
nielosowej}\label{selekcja-do-pruxf3by-nielosowej}

\begin{itemize}
\tightlist
\item
  Selekcja do próby nielosowej (\(S_A\), uwaga wielkość próby będzie
  losowa!):
\end{itemize}

\[
\text{logit}(\pi_{B,i}) = (3.5 + \mathbf{\alpha}^T(\log(\mathbf{X}_i)^2) - \sin(X_{3i} + X_{4i}) - X_{5i} - X_{6i})
\]

gdzie \(\mathbf{\alpha}=(0,0,0,3,-3,3,-3,0,...,0)^T\). Przykładowo
\(X_{3i}\) oznacza 3 kolumnę macierzy X.

\subsubsection{Selekcja do próby
losowej}\label{selekcja-do-pruxf3by-losowej}

\begin{itemize}
\tightlist
\item
  Selekcja do próby losowej (\(S_B\)) o liczebności 1,000:
\item
  losowanie proste z warstwowaniem
\item
  4 warstwy utworzone według następującej kombinacji 2 zmiennych:
\end{itemize}

\[
\begin{cases}
\text{Warstwa 1} & = X_2 < 0.5 & X_4 < 0,\\
\text{Warstwa 2} & = X_2 >= 0.5 & X_4 < 0,\\
\text{Warstwa 3} & = X_2 < 0.5 & X_4 >= 0,\\
\text{Warstwa 4} & = X_2 > 0.5 & X_4 >= 0,\\
\end{cases}
\]

\begin{itemize}
\item
  liczebność próby w poszczególnych warstwach wynosi: 100, 200, 300,
  400,
\item
  prawdopodobieństwo inkluzji dla każdej warstwy wynosi:
  \(\pi_w=(\frac{1}{100}, \frac{2}{100}, \frac{3}{100}, \frac{4}{100})\),
\item
  każdej wylosowanej jednostce należy przypisać wagę \(d_i\) będącą
  odwrotnością inkluzji warstwy, w której się znajduje.
\end{itemize}

\subsubsection{Procedura symulacji}\label{procedura-symulacji}

Procedurę należy powtórzyć 500 razy, w każdym przebiegu:

\begin{itemize}
\tightlist
\item
  należy zapisać średnią \(Y_1\) i \(Y_2\) z próby nielosowej \(S_A\),
\item
  należy zadeklarować obiekt \texttt{svydesign} dla próby losowej
  \(S_B\) uwzględniając argument \texttt{strata} określający warstwę.
\end{itemize}

\end{document}
